%        File: GraphDependencyParsing.tex
%     Created: 周一 十二月 05 04:00 下午 2011 W EST
% Last Change: 周一 十二月 05 04:00 下午 2011 W EST
%
\documentclass[slidestop,compress]{beamer}
\usetheme{Antibes}
%\usecolortheme{lily}
%\usepackage{CJK}
%\setbeameroption{show notes}
%\setbeameroption{show notes on second screen=right}
\usepackage{tikz}
\usetikzlibrary{arrows}
\usepackage{tikz-qtree}
\usepackage{amsmath}
\usepackage{tikz-dependency}
\usepackage{xunicode}
\usepackage{fontspec}
\usepackage{colortbl}
\usepackage{algorithmic}
\usepackage{tikz-qtree}
%\setsansfont{微软雅黑}
\DeclareMathOperator*{\argmax}{arg\,max}
\DeclareMathOperator*{\argmin}{arg\,min}
\title{Dialog Act tagging with CRF}
%\title{\Tree [.S [.NP LaTeX ] [.VP [.V is ] [.NP fun ] ] ]}
\author{Wei Qiu \and Natalia Korchagina \and Bruno Andrimiarina}
\date{Jan 15, 2013}
\begin{document}
\begin{frame}
    \titlepage
\end{frame}

\AtBeginSection[] % Do nothing for \section*
{
\begin{frame}<beamer>
\frametitle{Outline}
\tableofcontents[currentsection]
\end{frame}
}

\AtBeginSubsection[] % Do nothing for \subsection*
{
\begin{frame}<beamer>
\frametitle{Outline}
\tableofcontents[currentsection,currentsubsection]
\end{frame}
}

\frame{\frametitle{Outline}
\setcounter{tocdepth}{2}
\tableofcontents
}

%\section{Three graphical models}
%\subsection{Factored RBM}
%\subsection{Temporal Factored RBM}
%
%\frame{\frametitle{Motivation}
%\begin{itemize}
%    \item As in N-gram models, Markov assumption is still made in factored RBM.
%    \item If we want to use larger context, normally we will have more (exponentially) parameters to estimate.
%    \item take advantage large context without increasing dramatically number of parameters.
%\end{itemize}
%}
%
%\frame{\frametitle{Temporal Factored RBM}
%\begin{center}
%\includegraphics[width=0.8\textwidth]{TRBM.png}
%\end{center}
%}
%
%\frame{\frametitle{Turn FRBM into a temporal model}
%\begin{itemize}
%    \item Given a sequence $w_{1:t}$, apply an instance of the FRBM to each of the n-tuples in the sequence in succession.
%    \item Make the hidden units of the $n^{th}$ instance depend on the hidden units of the $n-1^{st}$ instance by making the hidden biases of the $n^{th}$ instance a linear function of the hidden states on the $n-1^{st}$ instance.
%    \item Make predictions as before, but use the new ``shifted'' biases.
%\end{itemize}
%}
%
%\frame{\frametitle{Learning and Inference in TFRBM}
%\begin{itemize}
%    \item Exact Inference in TFRBM is in tractable because there are too many variables depend on each other. 
%        \begin{itemize}
%            \item even filtering is intractable
%        \end{itemize}
%    \item Approximate filtering by using the mean field approximation to the previous hidden state distribution when shifting the biases.
%    \item Temporal connections are learned greedily by treating the previous hidden state as a constant put and using the CD learning rule.
%    \item The non-temporal parameters are learned as before.
%\end{itemize}
%}
%
%\subsection{Logbilinear model}
%\frame{\frametitle{Motivation}
%\begin{itemize}
%    \item It's still hard to learn with a bunch of hidden variables!
%    \item what about getting rid of them?
%    \item Assumption: the next word is kind of ``linear transformation'' of context words.
%\end{itemize}
%}
%
%\frame{\frametitle{Log-bilinear model}
%\begin{itemize}
%    \item Energy function: $E(w_{1:n}) = -\sum_{i=1}^{n-1} w_i^TRC_iR^Tw_n$
%    \item The resulting model can be viewed as FRBM with visible-to-visible connections but without hidden units.
%\end{itemize}
%}
%
%\frame{\frametitle{Log-bilinear model}
%\begin{center}
%\includegraphics[width=0.7\textwidth]{logbilib.png}
%\end{center}
%}
%
%\section{Conclusion}
%\frame{\frametitle{Conclusion}
%\begin{itemize}
%    \item Log-bilinear models outperform FRBM-based models as well as the best n-gram models and are easier to train than models with hidden units.
%    \item Adding temporal connections to the FRBM model makes it perform much better.
%    \item Averaging the predictions of any network model with a good n-gram model results in better predictions than using any model on its own.
%    \item \textcolor{red}{Remarks}: this paper is highly inspired by \textcolor{red}{deep Learning} which seems very promising.
%\end{itemize}
%}

\section{Introduction}
\frame{\frametitle{Introduction}
Dialog act tagging:
\begin{itemize}
    \item A module in modern dialog system
    \item guide the system to take certain action according to user's response
    \item a lot of researchers treat it as a partially observable Markov decision process.
\end{itemize}
}


\frame{\frametitle{Corpus analysis}
In this project we are working on a small part of corpus from ALLEGRO chatterbox.
Some statistics of this corpus:
\begin{itemize}
    \item Average lines per dialog: 35.0
    \item Average sentences per dialog: 65.0
    \item Average words per dialog: 376.0
    \item Average words per sentences: 5.0
\end{itemize}

}

\frame{\frametitle{Corpus analysis}
Every speech in the dialogs was tagged wrt the given annotation tags.
\begin{itemize}
    \item Request: ask(ready), ask(age), ask(gender), etc
    \item Answer: answer(ready), answer(no-ready)
    \item information, inform(table-objects), inform(cheese-where), etc
    \item Confirmation: confirt(nettoie\_sol, preferred\_option)
    \item Greeting: greet
    \item Acknowledge: ack
    \item Out of context: other
    \item Don't know: dont-know
\end{itemize}
}

\frame{\frametitle{Corpus analysis}
Problems we've encountered in the corpus:
\begin{itemize}
    \item Typo in the tags, ``possible\_option vs posible option'' which introduce more tags than predefined tags.
    \item some tags are too general, ``other''
    \item A lot of dialogs follow the same pattern, which would lead to overfitting.
    \item unbalanced, some tags are extremely rare.
\end{itemize}
}
\section{Dialog act tagging as sequential labeling}
\frame{\frametitle{Dialog act tagging as sequential labeling}
Dialog act tagging in real running system can not be treated as sequential labeling simply:
\begin{itemize}
    \item there is no observable data for system's current move
    \item the system only cares about the current state, it doesn't care about the mistakes made in the history.
\end{itemize}
But we can work around it by shifting the alignment.
}

\frame{\frametitle{Dialog act tagging as sequential labeling}
\begin{table}
    \begin{tabular}{|l|l|}
        \hline
        mana    & other              \\ \hline
        mana    & ask(age)           \\ \hline
        learner & answer(age)        \\ \hline
        mana    & ask(gender)        \\ \hline
        learner & answer(gender)     \\ \hline
        mana    & ask(firstLang)     \\ \hline
        learner & answer(firstLang)  \\ \hline
        mana    & ask(bilingue)      \\ \hline
        learner & answer(bilingue)   \\ \hline
        mana    & ask(secondLang)    \\ \hline
        learner & ?                  \\ \hline
    \end{tabular}
    \caption{To tag the content generated by user}
    \label{table:user_align}
\end{table}

}

\frame{\frametitle{Dialog act tagging as sequential labeling}
\begin{table}
    \begin{tabular}{|l|l|}
        \hline
        mana    & Begin \\ \hline
        mana    & other              \\ \hline
        learner & ask(age)           \\ \hline
        mana    & answer(age)        \\ \hline
        learner & ask(gender)        \\ \hline
        mana    & answer(gender)     \\ \hline
        learner & ask(firstLang)     \\ \hline
        mana    & answer(firstLang)  \\ \hline
        learner & ask(bilingue)      \\ \hline
        mana    & answer(bilingue)   \\ \hline
        learner & ask(secondLang)    \\ \hline
        End     & ?                  \\ \hline
    \end{tabular}
    \caption{To tag the system move}
    \label{table:system_align}
\end{table}

}
\frame{\frametitle{conditional random fields}
\begin{itemize}
    \item Similar to HMM
    \item but discriminative undirected graphical model 
    \item can make use of long distance features easily
    \item widely used in sequential labeling problem
    \item quite often provides state of the art result on sequential labeling task.

\end{itemize}
}

\section{Experiment and Result}
\frame{\frametitle{Features}
\begin{itemize}
    \item The agent of current speech
    \item Whether the current speech consists of multiple sentences.
    \item Whether the last sentence is a question or not
    \item *Bag of Words
        \begin{itemize}
            \item Stopwords are filtered
            \item tokenized by KEA
        \end{itemize}
\end{itemize}
We use CRF++ to train the model
}

\frame{\frametitle{Feature template for CRF++}
\begin{table}
    \begin{center}
    \begin{tabular}{|l|}
    \hline
    \#Unigram      \\ 
    U00:\%x[0,0]  \\ 
    U01:\%x[0,1]  \\ 
    U02:\%x[0,2]  \\ 
    U04:\%x[-1,0] \\ 
    U05:\%x[-1,1] \\ 
    U06:\%x[-1,2] \\ 
    \#Bigram       \\
    B\\
    \hline
    \end{tabular}
    \caption{Feature tempate for CRF++}
    \label{table:feature}
    \end{center}
\end{table}
}

\frame{\frametitle{Result}
\begin{itemize}
    \item Model without BoW features: 82.63\%
    \item Model with BoW features: 89.2\%
\end{itemize}
Some errors:
\begin{itemize}
    \item answer(bilingue)
    \item answer(nettoie\_sol, possible\_option)(!typo)
    \item answer(sencondlang)
    \item answer(dont\_know)
    \item reqRep
\end{itemize}
}

\section{Conclusion}
\frame{\frametitle{Conclusion}
\begin{itemize}
    \item we showed that even this task is not sequential labeling task, we can work around it easily
    \item BoW features are helpful for sequential labeling though it dramatically increase the complexity of the model.
    \item we can foresee that this method wouldn't work very well in a running system
    \item proposal for hierarchical classify. 
\end{itemize}
}
\end{document}


